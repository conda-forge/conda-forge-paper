%% Submissions for peer-review must enable line-numbering 
%% using the lineno option in the \documentclass command.
%%
%% Preprints and camera-ready submissions do not need 
%% line numbers, and should have this option removed.
%%
%% Please note that the line numbering option requires
%% version 1.1 or newer of the wlpeerj.cls file, and
%% the corresponding author info requires v1.2

\documentclass[fleqn,10pt,lineno]{wlpeerj} % for journal submissions
% \documentclass[fleqn,10pt]{wlpeerj} % for preprint submissions

\title{Conda-Forge: a community driven software packging ecosystem}

% Will fill in via the magic of jinja2 at some point
\author[1]{First Author}
\author[2]{Second Author}
\affil[1]{Address of first author}
\affil[2]{Address of second author}

% \keywords{Keyword1, Keyword2, Keyword3}

\begin{abstract}
Dummy abstract text. Dummy abstract text. Dummy abstract text. Dummy abstract text. Dummy abstract text. Dummy abstract text. Dummy abstract text. Dummy abstract text. Dummy abstract text. Dummy abstract text. Dummy abstract text.
\end{abstract}

\begin{document}

\flushbottom
\maketitle
\thispagestyle{empty}


\section*{Sketch}
\begin{enumerate}
\item Conda-Forge is a binary distro using conda aimed at making code 
available to users across platforms, operating systems, and languages.
\item Three perspectives:
\begin{enumerate}
\item Users: Use CF via `conda install xyz -c conda-forge`. CF focuses
on making sure that packages installed from CF are self consistant.
\item Package Maintainers: The CF workflow focuses on using the standard 
GitHub fork, branch, and pull request model. New packages are created via 
PRs into staged-recipes. Updates to existing packages are made by PRs into 
the associated feedstock.
\item Ecosystem Maintainers: Automated tools for keeping the ecosystem
self-consistant and up to date.
\end{enumerate}
\item Future work will focus on moving to new architectures (arm, powerpc,
web assembly) and reducing maintainer friction.
\end{enumerate}


\section*{Introduction}
\begin{enumerate}
\item Software is a critical part of science, with almost all research groups
using scientific software, and many groups writing software for their own and
other's use.
\item There is a gap (sometimes gulf) between writing software and using 
software. Packaging fills this gap.
\item Conda is a binary distribution, meaning that the difficulties associated
with compiling the code are handled by the code developers.
\item Conda-Forge is a packaging ecosystem based around the conda package 
manager which focuses on being a community-driven with high standards, a binary 
distribution with all the binaries built in the open, welcoming to beginners 
with support from packaging veterans.
\end{enumerate}

\section*{User's Perspective}
\begin{enumerate}
% why use conda/CF
\item Conda is a binary package manager.
\item CF's graph and SAT solver help maintain self-consistancy of
packages and environments
\item Adding the CF channel to the \texttt{.condarc} to enable CF
inside conda
\end{enumerate}

\section*{Package Maintainer's Perspective}
\begin{enumerate}
\item Submit new packages via GitHub using PRs into the staged-recipes repo 
[view graph of staged-recipes workflow]
\item Modify existing packages via PRs into feedstock repos [view graph of 
feedstock workflow]
\item CF uses continuous integration (CI) services to build the packages
\item Webservices provide interactive tools for linting and modification of PRs
\end{enumerate}

\section*{Ecosystem Maintainer's Perspective}
\begin{enumerate}
\item The CF graph allows us to rebuild the entirety of conda-forge in 
topological order [include view graph of graph]
\item The autotick bot keeps versions up to date and allows for the 
rebuilding of conda-forge
\item The global pinnings repo determines what the net ABI surface is, 
keeping the ecosystem consistant
\item The CF graph is accessable via \texttt{cf-graph3} and \texttt{libcfgraph}
\end{enumerate}


\section*{Conclusions}
\begin{enumerate}
\item CF provides a powerful platform for the distribution of scientific 
software focusing on ecosystem stability, build tracability, and automation.
\end{enumerate}


\section*{Acknowledgments}

Work in the Billinge group was supported by U.S. Department of Energy,
Office of Science, Office of Basic Energy Sciences (DOE-BES) under contract
No. DE-SC00112704.

\bibliography{sample}

\end{document}
