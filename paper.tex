%% Submissions for peer-review must enable line-numbering
%% using the lineno option in the \documentclass command.
%%
%% Preprints and camera-ready submissions do not need
%% line numbers, and should have this option removed.
%%
%% Please note that the line numbering option requires
%% version 1.1 or newer of the wlpeerj.cls file, and
%% the corresponding author info requires v1.2

\documentclass[fleqn,10pt,lineno]{wlpeerj} % for journal submissions
% \documentclass[fleqn,10pt]{wlpeerj} % for preprint submissions

\newcommand{\sjb}[1]{\textcolor{blue}{[sjb:#1]}}
\newcommand{\cjw}[1]{\textcolor{green}{[cjw:#1]}}

% Simon utils
\newcommand{\be}{\begin{enumerate}}
\newcommand{\ee}{\end{enumerate}}
\newcounter{saveenumi}
\newcommand{\seti}{\setcounter{saveenumi}{\value{enumi}}}
\newcommand{\conti}{\setcounter{enumi}{\value{saveenumi}}}


\title{Conda-Forge: a community driven software packaging ecosystem}

% Will fill in via the magic of jinja2 at some point
\author[1]{First Author}
\author[2]{Second Author}
\affil[1]{Address of first author}
\affil[2]{Address of second author}

% \keywords{Keyword1, Keyword2, Keyword3}

\begin{abstract}
Dummy abstract text. Dummy abstract text. Dummy abstract text. Dummy abstract text. Dummy abstract text. Dummy abstract text. Dummy abstract text. Dummy abstract text. Dummy abstract text. Dummy abstract text. Dummy abstract text.
\end{abstract}

\begin{document}

\flushbottom
\maketitle
\thispagestyle{empty}


\section*{Sketch}
\begin{enumerate}
\item Conda-Forge is a binary distro using conda aimed at making code
available to users across platforms, operating systems, and languages.
\item Three perspectives:
\begin{enumerate}
\item Users: Use CF via `conda install xyz -c conda-forge`. CF focuses
on making sure that packages installed from CF are self consistant.
\item Package Maintainers: The CF workflow focuses on using the standard
GitHub fork, branch, and pull request model. New packages are created via
PRs into staged-recipes. Updates to existing packages are made by PRs into
the associated feedstock.
\item Ecosystem Maintainers: Automated tools for keeping the ecosystem
self-consistant and up to date.
\end{enumerate}
\item Future work will focus on moving to new architectures (arm, powerpc,
web assembly) and reducing maintainer friction.
\end{enumerate}


\section*{Introduction}
\begin{enumerate}
\item Software is a critical part of modern life, from health to commerce to science.
\item It is increasingly important to be able to distribute and share software.
\item This is particularly important in the context of open-source community developed software.
\item for example in scientific software many groups write software for their own use, but want to share the code more broadly.
\item Packaging software and uploading to package repositories such as pypi~\cite{??} and conda~\cite{???} serves this role.
\item Conda hosts binary distributions, meaning that the difficulties associated
with compiling and releasing the code must be handled by the code developers themselves.
\item This is often a bottleneck as software developers are often domain experts but not necessarily expert in compilation and distribution of software
\item Conda-Forge addresses this need, by separating the code development and package management tasks.
\item Conda-Forge is a packaging ecosystem based around the conda package
manager.  
\item It focuses on being community-driven with high standards
\item It forms a natural partner with the open-source community software movement
\item It produces a binary
distribution but with all the binaries built in the open, in an environment that is welcoming to beginners with support from packaging veterans.
\item This paper describes the Conda Forge philosophy and organization and we hope it will
inspire people to join the conda-forge community.
\end{enumerate}\seti

\section*{User's Perspective}
\begin{enumerate}\conti
% why use conda/CF
\item Conda is a binary package manager, etc..
\item compare and contrast conda with pypi
\item CF and conda:  from the user perspective, conda-forge is a channel on conda
\item Why should a user set conda-forge as default channel?
\begin{enumerate}
\item CF has many of the commonly used packages, and is growing every day
\item CF packages are gaurenteed to interoperate with other CF packages
(which is a promise that not every channel can make)
\item it is easy to add packages to CF (if a user ever wanted to release their
own package)
\end{enumerate}
\item Setting up conda to use CF: Adding the CF channel to the \texttt{.condarc} to enable CF
inside conda
\end{enumerate}\seti

\section*{Package Maintainer's Perspective}
\begin{enumerate}\conti
\item Building packages for all platforms and archetactures is hard, requireing
a lot of infrastructure and hardware
\item CF provides this infrastructure via continuous integration (CI) services,
allowing builds on Windows, Mac, and Linux platforms.
\item CF uses GitHub to hold the package recipes, the instructions for how to 
build the package, and the CI configuration files in repositories called ``feedstocks''
%\item probably need something about how package management is done in conda but outside of CF, so that when CF is described it looks much better than this.
\item Workflows:
\begin{enumerate}
    \item Submitting a new package
        \begin{enumerate}
            \item Fork \texttt{conda-forge/staged-recipes}
            \item Create a new branch
            \item Add the recipe in a new folder under \texttt{recipes}
            \item Open a pull request (PR) with the new recipe
            \item Once the PR has been merged, a new feedstock repo
                will be created and the folder removed
%\sjb{turn this around.  CF has a repo on GitHub called staged-recipes where package requirements are maintained.  New packages are added to staged-recipes via usual GitHub PR workflow.}
[view graph of staged-recipes workflow]
        \end{enumerate}
    \item Modifying an existing package
        \begin{enumerate}
            \item Fork the feedstock
            \item Create a new branch
            \item Change the existing recipe
            \item Open a pull request (PR) with the changes
            \item Once the PR has been merged, the package will be updated
[view graph offeedstock workflow]
        \end{enumerate}
\end{enumerate}
%\sjb{see above for the more gentle way of introducing these ideas}
\item Webservices provide interactive tools for linting and modification of PRs
\end{enumerate}\seti

\section*{Ecosystem Maintainer's Perspective}
\begin{enumerate}\conti
\item The CF graph allows us to rebuild the entirety of conda-forge in
topological order [include view graph of graph]
\item One of the most common changes to packages is version bumps, which are
tedious to produce by hand, and were often out of sync with the source code
version changes, since recipe mantianers aren't nessisarally aware of 
changes in all the packages they maintain.
\item The autotick bot was designed to address this issue by searching for
updates to the source code of all the packages that conda-forge manages,
and issuing PRs for packages which need to be updated.
\item Maintaining self consistancy is especially important for binary packages.
Binary packages ship an Application Binary Interface (ABI) which describes how other software
can interact with the compiled binary.
\item Mixing ABIs can be disastarous as the mismatch in expected ABI and 
the actual ABI produces difficult to debug errors or erronious results.
\item Without self consistancy packages could depend on different ABIs
producing situations where packages are sighlently incompatable, forcing
users to choose which part of the ecosystem they can use.
\item To address this issue, and to make certain that each package in the 
ecosystem is compatable with the rest of the ecosystem CF has a global 
pinnings repository.
This repository lists the current version of the package which CF builds
all of its packages against, effectively defining all the ABIs that CF ships.
\item The autotick bot infrastructure also helps maintain compatability by
allowing for large scale migrations. These migrations move the entire ecosystem
from one ABI version to another, rebuilding all the packages in dependency
order.
\item The CF graph is accessable via \texttt{cf-graph3} and \texttt{libcfgraph}.
These repos allow users to query the graph and artifacts
\end{enumerate}\seti


\section*{Conclusions}
\begin{enumerate}\conti
\item CF provides a powerful platform for the distribution of scientific
software focusing on ecosystem stability, build tracability, and automation.
\item users: use it
\item software developers (and others): consider becoming CF contributors
\end{enumerate}\seti


\section*{Acknowledgments}

Work in the Billinge group was supported by U.S. Department of Energy,
Office of Science, Office of Basic Energy Sciences (DOE-BES) under contract
No. DE-SC00112704.

\bibliography{sample}

\end{document}
