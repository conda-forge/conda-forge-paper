%% Submissions for peer-review must enable line-numbering
%% using the lineno option in the \documentclass command.
%%
%% Preprints and camera-ready submissions do not need
%% line numbers, and should have this option removed.
%%
%% Please note that the line numbering option requires
%% version 1.1 or newer of the wlpeerj.cls file, and
%% the corresponding author info requires v1.2

\documentclass[fleqn,10pt,lineno]{wlpeerj} % for journal submissions
% \documentclass[fleqn,10pt]{wlpeerj} % for preprint submissions

\newcommand{\sjb}[1]{\textcolor{blue}{[sjb:#1]}}

% Simon utils
\newcommand{\be}{\begin{enumerate}}
\newcommand{\ee}{\end{enumerate}}
\newcounter{saveenumi}
\newcommand{\seti}{\setcounter{saveenumi}{\value{enumi}}}
\newcommand{\conti}{\setcounter{enumi}{\value{saveenumi}}}


\title{Conda-Forge: a community driven software packaging ecosystem}

% Will fill in via the magic of jinja2 at some point
\author[1]{First Author}
\author[2]{Second Author}
\affil[1]{Address of first author}
\affil[2]{Address of second author}

% \keywords{Keyword1, Keyword2, Keyword3}

\begin{abstract}
Dummy abstract text. Dummy abstract text. Dummy abstract text. Dummy abstract text. Dummy abstract text. Dummy abstract text. Dummy abstract text. Dummy abstract text. Dummy abstract text. Dummy abstract text. Dummy abstract text.
\end{abstract}

\begin{document}

\flushbottom
\maketitle
\thispagestyle{empty}


\section*{Sketch}
\begin{enumerate}
\item Conda-Forge is a binary distro using conda aimed at making code
available to users across platforms, operating systems, and languages.
\item Three perspectives:
\begin{enumerate}
\item Users: Use CF via `conda install xyz -c conda-forge`. CF focuses
on making sure that packages installed from CF are self consistant.
\item Package Maintainers: The CF workflow focuses on using the standard
GitHub fork, branch, and pull request model. New packages are created via
PRs into staged-recipes. Updates to existing packages are made by PRs into
the associated feedstock.
\item Ecosystem Maintainers: Automated tools for keeping the ecosystem
self-consistant and up to date.
\end{enumerate}
\item Future work will focus on moving to new architectures (arm, powerpc,
web assembly) and reducing maintainer friction.
\end{enumerate}


\section*{Introduction}
\begin{enumerate}
\item Software is a critical part of modern life, from health to commerce to science.
\item It is increasingly important to be able to distribute and share software.
\item This is particularly important in the context of open-source community developed software.
\item for example in scientific software many groups write software for their own use, but want to share the code more broadly.
\item Packaging software and uploading to package repositories such as pypi~\cite{??} and conda~\cite{???} serves this role.
\item Conda hosts binary distributions, meaning that the difficulties associated
with compiling and releasing the code must be handled by the code developers themselves.
\item This is often a bottleneck as software developers are often domain experts but not necessarily expert in compilation and distribution of software
\item Conda-Forge addresses this need, by separating the code development and package management tasks.
\item Conda-Forge is a packaging ecosystem based around the conda package
manager.  
\item It focuses on being community-driven with high standards
\item It forms a natural partner with the open-source community software movement
\item It produces a binary
distribution but with all the binaries built in the open, in an environment that is welcoming to beginners with support from packaging veterans.
\item This paper describes the Conda Forge philosophy and organization and we hope it will
inspire people to join the conda-forge community.
\end{enumerate}\seti

\section*{User's Perspective}
\begin{enumerate}\conti
% why use conda/CF
\item Conda is a binary package manager, etc..
\item compare and contrast conda with pypi
\item CF and conda:  from the user perspective, conda-forge is a channel on conda
\item Why should a user set conda-forge as default channel?
\item CF's graph and SAT solver help maintain self-consistancy of
packages and environments\sjb{does the user care about this?  Should this be in the next section?  Or if here, motivate why a user would care about this.}
\item Setting up conda to use CF: Adding the CF channel to the \texttt{.condarc} to enable CF
inside conda
\end{enumerate}\seti

\section*{Package Maintainer's Perspective}
\begin{enumerate}\conti
\item probably need something about how package management is done in conda but outside of CF, so that when CF is described it looks much better than this.
\item The CF approach: Submit new packages via GitHub to ??? using PRs into the staged-recipes repo\sjb{turn this around.  CF has a repo on GitHub called staged-recipes where package requirements are maintained.  New packages are added to staged-recipes via usual GitHub PR workflow.}
[view graph of staged-recipes workflow]
\item Modify existing packages via PRs into feedstock repos [view graph of
feedstock workflow]\sjb{see above for the more gentle way of introducing these ideas}
\item CF uses continuous integration (CI) services to build the packages
\item Webservices provide interactive tools for linting and modification of PRs
\end{enumerate}\seti

\section*{Ecosystem Maintainer's Perspective}
\begin{enumerate}\conti
\item The CF graph allows us to rebuild the entirety of conda-forge in
topological order [include view graph of graph]
\item The autotick bot keeps versions up to date and allows for the
rebuilding of conda-forge\sjb{again, start with the problem, then how CF solves the problem.}
\item The global pinnings repo determines what the net ABI surface is,
keeping the ecosystem consistant\sjb{no idea what htis means, so it also definitely needs to be broken out into a series of items that describe the problem and the CF solution.}
\item The CF graph is accessable via \texttt{cf-graph3} and \texttt{libcfgraph}
\end{enumerate}\seti


\section*{Conclusions}
\begin{enumerate}\conti
\item CF provides a powerful platform for the distribution of scientific
software focusing on ecosystem stability, build tracability, and automation.
\item users: use it
\item software developers (and others): consider becoming CF contributors
\end{enumerate}\seti


\section*{Acknowledgments}

Work in the Billinge group was supported by U.S. Department of Energy,
Office of Science, Office of Basic Energy Sciences (DOE-BES) under contract
No. DE-SC00112704.

\bibliography{sample}

\end{document}
